\documentclass[a4paper,12pt]{article} %style de document
\usepackage[utf8]{inputenc} %encodage des caractères
\usepackage[french]{babel} %paquet de langue français
\usepackage[T1]{fontenc} %encodage de la police
\usepackage{times}
\usepackage[top=2cm,bottom=2cm,left=2cm,right=2cm]{geometry} %marges
\usepackage{graphicx} %'affichage des images
\usepackage{enumitem}
\usepackage{hyperref}
\usepackage{fancyhdr}
\pagestyle{fancy}
\usepackage{amssymb}
\usepackage[useregional]{datetime2}
\usepackage{datetime}
\usepackage{listings}
\newdateformat{monthyeardate}

\renewcommand{\footrulewidth}{1pt}
\fancyfoot[R]{\textbf{page \thepage}}
\fancyfoot[C]{}
\fancyfoot[L]{Travail Personnel}
\renewcommand{\headrulewidth}{0pt}

\usepackage{verbatim}

\author{Guillaume LEMONNIER}

\title{Ascii Image\\Projet Personnel}

\begin{document}

\maketitle

\newpage

\tableofcontents

\newpage

\section{Introduction}

\subsection{Raison de la création du Projet}

J'ai eu cette idée quand j'ai regarder une image en ascii et j'ai voulu en faire ... Voila

\subsection{Première approche}

\newpage

\section{Vision du projet}

\subsection{Fonctions codés (pseudo langage)}

\newpage

\section{Codage}

\subsection{Structures de données}

\subsubsection{Name 1}

\end{document}